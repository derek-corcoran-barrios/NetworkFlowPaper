\documentclass[]{article}
\usepackage{lmodern}
\usepackage{amssymb,amsmath}
\usepackage{ifxetex,ifluatex}
\usepackage{fixltx2e} % provides \textsubscript
\ifnum 0\ifxetex 1\fi\ifluatex 1\fi=0 % if pdftex
  \usepackage[T1]{fontenc}
  \usepackage[utf8]{inputenc}
\else % if luatex or xelatex
  \ifxetex
    \usepackage{mathspec}
  \else
    \usepackage{fontspec}
  \fi
  \defaultfontfeatures{Ligatures=TeX,Scale=MatchLowercase}
\fi
% use upquote if available, for straight quotes in verbatim environments
\IfFileExists{upquote.sty}{\usepackage{upquote}}{}
% use microtype if available
\IfFileExists{microtype.sty}{%
\usepackage{microtype}
\UseMicrotypeSet[protrusion]{basicmath} % disable protrusion for tt fonts
}{}
\usepackage[margin=1in]{geometry}
\usepackage{hyperref}
\hypersetup{unicode=true,
            pdftitle={Target-based site prioritization under climate change using quadratic network flow},
            pdfauthor={Derek Corcoran},
            pdfborder={0 0 0},
            breaklinks=true}
\urlstyle{same}  % don't use monospace font for urls
\usepackage{longtable,booktabs}
\usepackage{graphicx,grffile}
\makeatletter
\def\maxwidth{\ifdim\Gin@nat@width>\linewidth\linewidth\else\Gin@nat@width\fi}
\def\maxheight{\ifdim\Gin@nat@height>\textheight\textheight\else\Gin@nat@height\fi}
\makeatother
% Scale images if necessary, so that they will not overflow the page
% margins by default, and it is still possible to overwrite the defaults
% using explicit options in \includegraphics[width, height, ...]{}
\setkeys{Gin}{width=\maxwidth,height=\maxheight,keepaspectratio}
\IfFileExists{parskip.sty}{%
\usepackage{parskip}
}{% else
\setlength{\parindent}{0pt}
\setlength{\parskip}{6pt plus 2pt minus 1pt}
}
\setlength{\emergencystretch}{3em}  % prevent overfull lines
\providecommand{\tightlist}{%
  \setlength{\itemsep}{0pt}\setlength{\parskip}{0pt}}
\setcounter{secnumdepth}{5}
% Redefines (sub)paragraphs to behave more like sections
\ifx\paragraph\undefined\else
\let\oldparagraph\paragraph
\renewcommand{\paragraph}[1]{\oldparagraph{#1}\mbox{}}
\fi
\ifx\subparagraph\undefined\else
\let\oldsubparagraph\subparagraph
\renewcommand{\subparagraph}[1]{\oldsubparagraph{#1}\mbox{}}
\fi

%%% Use protect on footnotes to avoid problems with footnotes in titles
\let\rmarkdownfootnote\footnote%
\def\footnote{\protect\rmarkdownfootnote}

%%% Change title format to be more compact
\usepackage{titling}

% Create subtitle command for use in maketitle
\providecommand{\subtitle}[1]{
  \posttitle{
    \begin{center}\large#1\end{center}
    }
}

\setlength{\droptitle}{-2em}

  \title{Target-based site prioritization under climate change using quadratic network flow}
    \pretitle{\vspace{\droptitle}\centering\huge}
  \posttitle{\par}
    \author{Derek Corcoran}
    \preauthor{\centering\large\emph}
  \postauthor{\par}
    \date{}
    \predate{}\postdate{}
  
\usepackage{booktabs}
\usepackage{longtable}
\usepackage{array}
\usepackage{multirow}
\usepackage{wrapfig}
\usepackage{float}
\usepackage{colortbl}
\usepackage{pdflscape}
\usepackage{tabu}
\usepackage{threeparttable}
\usepackage{threeparttablex}
\usepackage[normalem]{ulem}
\usepackage{makecell}
\usepackage{xcolor}

\begin{document}
\maketitle

\hypertarget{introduction}{%
\section{Introduction}\label{introduction}}

Currently, 14.9\% of terrestrial area is a part of the global protected area network (UNEP-WCMC, 2018). However, most protected areas have not been created using priorization tools developed to maximize outcomes such as species or ecosystems conservation based on current biodiversity patterns, resulting in underperforming conservation networks globally (Rodrigues et al., 2004; Jenkins, Van Houtan, Pimm, \& Sexton, 2015). The situation is even worst when we take into account that species distributions are expected to change over time as a result of anthropogenic climate change (Araújo, Cabeza, Thuiller, Hannah, \& Williams, 2004; Lenoir, Gégout, Marquet, De Ruffray, \& Brisse, 2008; Chen, Hill, Ohlemüller, Roy, \& Thomas, 2011; Regos et al., 2016), if species that are now protected migrate out of protected areas in the future (Araújo et al., 2004; Hannah et al., 2007; Alagador, Cerdeira, \& Araújo, 2014; Marquet, Lessman, \& Shaw, 2019).

Different approaches and workflows have been recently developed with the aim to inform conservation planning in adapting conservation networks to climate change (Marquet et al., 2019). One of the main avenues of research focuses on finding areas that are able to include both species current and future projected distributions. However, a review by Jones, Watson, Possingham, \& Klein (2016) showed that more than 90\% of the prioritization planning studies following this approach did only consider biodiversity needs in two time steps: where does biodiversity occur under current climatic conditions and where is it projected to occur at a certain year ahead in the future (typically, at the end of the century). The concern is that these solutions may be insufficient to effectively allow species meet their future suitable ranges from their current distributions, or that solutions are not adequately optimized. A major problem preventing more adequate solutions is methodological limitation. Currently, the most used tools for conservation planning are Marxan, Prioritizr and Zonation (Di Minin et al., 2014; Hanson et al., 2019). From these, Zonation includes more advanced options to prioritize conservation under climate change scenarios, as conservation features may be explicitly linked through interaction (i.e.: Reside, VanDerWal, \& Moran (2017)). This allows for simultaneous prioritization of a species current range, its modelled future range, and the connectivity between the two limited by a species' capacity to disperse, but it is restricted to using one future modelled range per species. On the other hand, Prioritizr and Marxan do not include specific options to consider climate change related biodiversity needs.

\begin{table}[H]
\centering
\begin{tabular}{lllll}
\toprule
 & Network flow & Zonation & Migclim & Prioritizr\\
\midrule
Considers more than 2 time-slices & Yes & No & Yes & No\\
Can add species to the solutions afterwards & Yes & No & Yes & No\\
Keep track of species dispersal routes & Yes & Yes & Yes & No\\
Solves for conservation targets & Yes & Yes & No & Yes\\
Considers cost as part of the solution & Yes & Yes & No & Yes\\
\bottomrule
\end{tabular}
\end{table}

However, in order for the species to move from areas were they are currently protected to the future protected areas, their route has to be protected as well, that is the biological corridors that they will use (Rosenberg, Noon, \& Meslow, 1997; Nuñez et al., 2013). These routes might be referred as climate biological corridors. The planning and creation of these biological corridors correspond to adaptive measures to environmental changes that would mitigate the impacts of the climate crisis (Hannah et al., 2007). In that context, biological corridors should consider the progressivity of the effects of the climate crisis, including changes in suitable climatic conditions and in land use and habitat fragmentation, in order to evaluate the feasibility of the species reaching their future available habitat.

There have been several approaches trying to model biological corridors. Local approaches have focus just in one or a few species (Cushman et al., 2013; Alagador et al., 2014; Gregory et al., 2014) or group of species (Williams et al., 2005; Beier, Majka, \& Bayless, 2007; Phillips, Williams, Midgley, \& Archer, 2008), not being able to be use for planning for lack of species or be in confined spaces, respectively. A continental approach was made by Lawler, Ruesch, Olden, \& McRae (2013), who modelled biological corridors in the Americas. Even when that model was made for close to 3,000 species, but it did not consider species dispersal speed, and it did not use the information on species migration routes to identify priority areas that are climatically resilient.

Among these articles, the work of Phillips et al. (2008), incorporates the use of the Network Flow optimization method (Ahuja, Magnanti, \& Orlin, 1993) in order to solve the problem expressed by Williams et al. (2005), generating a solution 33\% more efficient than previous optimization methods. This makes this methodology one of the most promising ways of solving conservation planning problems through the use of biological corridors.

In its traditional use, Network flow gives the best possible solution given constrains (Phillips et al., 2008). If we looked at the solution for a species in a raster, every cell would have a value of either one or zero, where one would be a cell necessary for the conservation of the species or a zero if it is not necessary. This best solution, however, only represents one of the possible solutions to conserve a species, and does not give us any alternatives. in this research we present Quadratic Cost Network Flow as alternative solution, where for every species we would still get a value of one if the cell is irreplaceable for the solution, zero if that cell is not needed, and an intermediate value for cells that are one of many alternative solutions for a given species, this is, for every cell we have an irreplaceability index. Spatial planning is a complex endeavour where agendas and/or needs evolve rapidly and several alternative suboptimal solutions might be needed in order to navigate this complex decisions.

In this article we present a new method of working with network flow on conservation planning using quadratic costs, while at the same time comparing its results and implications with traditional network flow and zonation, using the Northern Tropical Andes ecorregion as an example.

\hypertarget{methods}{%
\section{Methods}\label{methods}}

\hypertarget{area}{%
\subsection{Area}\label{area}}

\begin{figure}

{\centering \includegraphics{NFPaper_files/figure-latex/MapArea-1} 

}

\caption{Map of Southamerica, the darkened area is the area where the priorization was performed}\label{fig:MapArea}
\end{figure}

The studied area was the Northern Tropical Andes, including the countries of Colombia, Venezuela and Ecuador (Figure \ref{fig:MapArea}). To avoid pixels on countries' borders receiving a lower priority due to a boundary effect, we extended the analyses to a buffer of 200 kilometers around their terrestrial domains. The study area encompasses a total of 3,632,762 square kilometres.

\hypertarget{global-circulation-models}{%
\subsection{Global circulation models}\label{global-circulation-models}}

\begin{figure}
\includegraphics[width=4.39in]{Diag1} \caption{All the steps and programs used in the Optimization}\label{fig:Diagrama}
\end{figure}

Global Circulation Models (GCMs) are predictive models that use fluid models\ldots{}
In this study, four GCM models were selected to project species distribution using the GCMcompareR shiny app (Fajardo, Corcoran, Roehrdanz, Hannah, \& Marquet, 2018), we followed the story-lines approach (Zappa \& Shepherd, 2017) and selected a GCM warmer and wetter, warmer and dryer, colder and wetter, and colder and dryer than the ensemble. The selected models where cc, cn, la, mp all the process is shown in in figure \ref{fig:Diagrama}.

\hypertarget{species-selection}{%
\subsection{Species selection}\label{species-selection}}

671 species of plants and vertebrates were used for the model, only endemic species were used in this study. A list can be found in the supplementary materials. The presences for plants and vertebrates were gather using different databases and methodologies.

For plants, all presences were collected from the BIEN database. The BIEN database is a product of a workflow of several tools, including two softwares: Taxonomic Name Resolution Service (Boyle et al., 2013) and Geographic Name Resolution Service (Boyle, 2019). The first resolves and corrects sinonymous taxa. The second validates, eliminates or flags coordinates accordingly. In this research, only records where the scientific name was clear to the species level, passed all geographic validations, was certain to be a native species, and was not part of a plot were included.

Vertebrates ocurrences were extracted form GBIF, Vertnet, and Global Restricted range bird species data bases. We only used presences from 1950 onwards, that were recorded by human observation. All centroids and longitudes or latitudes of exactly zero were removed. Points were also removed if they were further than 500 Km from the ranges developed by IUCN.

\hypertarget{species-distribution-models.}{%
\subsection{Species Distribution Models.}\label{species-distribution-models.}}

The Dismo Package (Hijmans, Phillips, Leathwick, \& Elith, 2017) was used to generate Species Distribution Models with the Maxent algorithm (Phillips, Anderson, \& Schapire, 2006; Merow, Smith, \& Silander Jr, 2013). We used the specifications recomended by Merow et al. (2013) and Merow et al. (2014) to generate models with less overfitting and with less complex responses to variables.
For the backround points we used 50,000 points within a polygon which encompassed all presences plus a 500 Km buffer. We used five replications of models for each species and used 30\% of the presences to test the model.

Environmental predictor variables were extracted from Worldclim. The 19 bioclimatic variables were checked for colinearity. If two variables were hightly correlated (over xx\% ), the simpler variable was always selected. Variables that considered precipitation only or temperature only, were always chosen over variables that combined both. When variables average over time, the shortest time where always prefered over longer time periods (e.g.~Minimum temperature of the coldest month would be chosen over minimum temperature of the coldest quarter). The selected layers were: Mean annual temperature (BIO1), mean diurnal temperature range (BIO2), seasonality of temperature (BIO4), Minimum temperature of the coldest month (BIO6), mean annual precipitation (BIO12), and seasonality of precipitation (BIO15).

An accumulated aridity index was used in an attempt to consider water defficit and dry season length. To calculate this index, we first find the longest stretch of months where potential evapotranspiration is larger than precipitation. Then, we sum the substraction of precipitation and potential evapotranspiration over this period of time.

Additionaly, five soil variables from Soilgrids (De Sousa, Heuvelink, Batjes, \& Kempen, 2018) were used (depth to bedrock, pH, clay proportion, silt proportion, and bulk density). If one of the variables had multiple strata, we only used the average of the first meter.

\hypertarget{cost-layer}{%
\subsection{Cost layer}\label{cost-layer}}

We used a 2.5 minutes downscaled version of Naidoo \& Iwamura (2007) layer of costs in order to estimate the cost per hectare for each area. To generate a cost layer to use for all prioritization algoritms, we modified that layer using the World Database on Protected Areas (WDPA) (UNEP-WCMC, 2018) and land-cover (Tuanmu \& Jetz, 2014): if a cell of the cost layer was inside a proteced area, this was multiplied by zero; if it was outside of a protected area, it was multiplied by one. We only considered protected areas of classes I and II of (WDPA). By doing this, we were able to consider with a cost of zero those areas that are already in the global network of protected areas. Furthermore, every cell that had over 50\% of urbanization or agriculture according to Tuanmu \& Jetz (2014) was taken out of the possible solutions. Our aim with this was to exclude areas that are heavily urbanized or intesely used for agriculture and might not be suitable for protection.

\hypertarget{prioritization-algorithms}{%
\subsection{Prioritization algorithms}\label{prioritization-algorithms}}

\hypertarget{network-flow}{%
\subsubsection{Network Flow}\label{network-flow}}

This model was developed based on the notion of non-overlapping dispersal chains (Williams et al., 2005) and commonly linear cost was used (Phillips et al., 2008). We used two aproaches in this study, the linear and quadratic Network Flow. For each species, linear network flow generates cells to be protected (values of one) and cells not to be protected (values of zero). On the other hand, quadratic network flow presents for each cell an irreplaceability index going from zero to one. Here, zero means that there are no chains passing by that spatial point that go from the source to the final destination, and one is a point where chains have to pass through to reach the desired flow.

\hypertarget{model-formulation}{%
\paragraph{Model Formulation}\label{model-formulation}}

The objective is to minimize the cost

\(\underset{Quadcost}{\text{minimize}}\)

\[\small Quadcost =  \sum_{i=1}^n{(Cost\times Flow_t)^2}\]

\(\text{subject to}\)

\[\small \text{Initial Flow} = \sum_{i=1}^n{Flow_{t_1}} = \text{Target paths}\]

\[\small \text{Final Flow} = \sum_{i=1}^n{Flow_{t_{final}}} = \text{Target paths}\]

\[\small \text{Flow capacity} = \sum_{i=1}^n{Flow_{(i,t)}} \leq capacity_{(i,t)}\]

\hypertarget{network-generation}{%
\subsubsection{Network generation}\label{network-generation}}

In order to generate the networks de QuadCostAmpl package was used (Corcoran \& Fajardo, 2019)

\hypertarget{linear-network-flow}{%
\paragraph{Linear Network Flow}\label{linear-network-flow}}

Model

\hypertarget{quadratic-network-flow}{%
\paragraph{Quadratic Network Flow}\label{quadratic-network-flow}}

In order to do this, we follow Network-flow methods (Ahuja et al., 1993) to model the survival and dispersion of species along the space and time, using species distribution models and quadratic flow costs to split the flow evenly across alternative paths. Therefore, the resulting flow across an arc is high only if there are few alternatives to use that edge to achieve the target flow.

\hypertarget{zonation}{%
\subsubsection{Zonation}\label{zonation}}

Prioritization under Climate Change. Zonation Conservation Planning Software is a widely used tool that allows for simultaneous prioritization of many thousands of conservation features (i.e.~species or ecosystem types) while also considering fundamental aspects of the landscape including habitat condition, connectivity, and cost (68-69). Zonation can be used for conservation prioritization under climate change as conservation features may be explicitly linked through interaction. This allows for simultaneous prioritization of a species current range, its modeled future range, and the connectivity between the two limited by a species' capacity to disperse. The protocol for running Zonation under uncertain climate change projections is thoroughly described in Kujala et al.~2013 (70).

Species current ranges were linked to projected future ranges through an interaction layer. This layer is transformed by a dispersal kernel with a parameter to limit the interaction to the species total capacity to disperse over the period of analysis. Total dispersal capacity was assumed to be 100km for vertebrates (roughly 1 km/yr) and 1km for vascular plants (roughly 100m/yr). The algorithm favors areas of the species range in baseline climate that are retained in the projected future climate (i.e.~the areas are suitable for species in both baseline and future climate). Prioritization were conducted using the mean projected future species ranges across 10 GCM under the same RCP (RCP2.6 or RCP8.5) that were discounted by the standard deviation the projected ranges. Prioritizations were also run for each GCM individually for both RCP2.6 and RCP8.5

Conservation Features. All available species models (vascular plants and vertebrates) were used in continental scale prioritizations. Prioritization used the continuous value output of species distribution models for current climates and projected future climates. See species distribution modeling section above for more information. Models were fit at 30 arc-second resolution and were projected into 5km resolution environmental layers. All available species models (vascular plants and vertebrates) were used in continental scale prioritizations. Species with too few occurrence records to produce a model were included as point locations (Zonation term = ``Species of Special Interest''). Equal weighting was used for all conservation features.

\hypertarget{prioritization-methods.}{%
\subsubsection{Prioritization Methods.}\label{prioritization-methods.}}

Zonation ranks all cells in a landscape according to their value and removes cells of lowest value at each processing step. Values are then recalculated for all remaining cells at each step. Core area zonation removal algorithm was used this assigns value to a cell from the highest value species present in that cell at each step of removal. Species with smaller ranges have few cells that are critical for their conservation these cells therefore tend to have high values. By contrast, wide ranging species have more opportunities to be conserved -- so are typically lower value until only the core range remains. Warp factor (number of cells removed at each step) for all analysis was set to 1 to maximize the accuracy of the rank order assigned. All prioritizations used edge removal and boundary length penalty of 1.

\hypertarget{comparizon-of-prioritization-methods}{%
\subsection{Comparizon of prioritization methods}\label{comparizon-of-prioritization-methods}}

\hypertarget{spearman-correlation}{%
\subsubsection{Spearman correlation}\label{spearman-correlation}}

We used pairwise Spearman correlation in order to compare the ranking of the cells between the different prioritization methods raster results (Zar, 1972).

\hypertarget{local-spearman-correlation}{%
\subsubsection{Local Spearman correlation}\label{local-spearman-correlation}}

We also tested for local correlation Spearman correlation,

\hypertarget{comparison}{%
\subsubsection{10\% comparison}\label{comparison}}

\hypertarget{results}{%
\section*{Results}\label{results}}
\addcontentsline{toc}{section}{Results}

\begin{figure}
\centering
\includegraphics{NFPaper_files/figure-latex/AllSols-1.png}
\caption{\label{fig:AllSols}Coninuous results of the three algorithms, the highest the value the higher the priority, all values go from 0 to 1.}
\end{figure}

\begin{itemize}
\tightlist
\item
  \textbf{For methods:} Since one of the main objectives of priorization methods is to rank the cells within an area, we used Spearman's rho correlation, that estimates the rank-based measure of association. We also calculated a local Spearman correlation on a matrix of five by five cells, in order to figure out if there are some important
\end{itemize}

There are two ways of trying to look at the results, as a continuous layer, as in figure \ref{fig:AllSols}, or as a binary solution as in fig \ref{fig:Binary}, where the cells in red are the top 10\% ranked cells to be protected. At first glance, the three algorithms have very similar results. In all three methods the Andes mountains appear to have the highest priorities in particular the Northwest Andean montane forests, Eastern Cordillera Real montane forests, Venezuelan Andes montane forests, Cordillera Oriental montane forests, and the Cauca and Magdalena Valley montane forests, outside of the Andes mountains the ecorregions that have the highest priorities are Guianan Highlands moist forests, Pantepui forests \& shrub-lands and Cordillera La Costa montane forests. Furthermore, when we look at the top 10\% cells selected by the algorithms (Figures \ref{fig:Binary} and \ref{fig:NumberOfTimes}), we see that of all the cells that were selected at least by one algorithm, 62.91\% of them were selected by all three algorithms, which show the level of concordance between cells, 18.85\% were selected by two of the algorithms and only 18.24\% of the cells selected were chosen by only one of the methods.

However, when we look at the Spearman correlation (table \ref{tab:Corr}), of all three methods we start seeing some important differences, where Quadratic and Linear Network flow have a spearman correlation of 0.92, but both methods have a lower correlation with zonation, with Quadratic Network Flow having a higher correlation of 0.669, vs linear network flow having a correlation of 0.624 with Zonation, all these relationships are significative. When we look at the local correlation plots (figure \ref{fig:LocalCorr}), we see that for the most part, the local correlations are all mostly positive, specially between both Network Flow algorithms. There are some blue areas in the figure, when comparing those priorities with zonation. Representing some big discrepancies in local prioritization, that is mostly localized

when we look at the box-plot of local correlation values (figure \ref{fig:Boxplot}), we see that the lower whiskers in the comparison of both Network Flow methods with Zonation, are bellow zero

\begin{itemize}
\tightlist
\item
  \textbf{For discussion:} One of the main problems at comparing Network Flow vs Zonation, is that Zonation will give a rank to each and every cell in the study area, which is not true for Network Flow. It comes as no surprise that Quadratic Network flow has a higher correlation with Zonation than Linear Network flow, the intended result of quadratic network flow being more similar to zonation than linear network flow.
\end{itemize}

\hypertarget{efficiency-comparison-number-of-cells}{%
\subsubsection{Efficiency comparison (Number of cells)}\label{efficiency-comparison-number-of-cells}}

\begin{figure}
\centering
\includegraphics{NFPaper_files/figure-latex/Binary-1.png}
\caption{\label{fig:Binary}Binary solution for the cells with the top 10\% value of ranked priority}
\end{figure}

In this example when we calculate the top 10\% of protected cells for Quadratic Network Flow we get 22,671 cells, 22,586 cells for linear Network Flow, and 22,671 cells for Zonation.

\begin{figure}
\centering
\includegraphics{NFPaper_files/figure-latex/NumberOfTimes-1.png}
\caption{\label{fig:NumberOfTimes}This map shows how many algorithms selected each cell}
\end{figure}

\hypertarget{correlation-comparison-spearman}{%
\subsubsection{Correlation comparison (Spearman)}\label{correlation-comparison-spearman}}

\begin{table}

\caption{\label{tab:Corr}Spearman correlation between the rank of the cells of all three methods}
\centering
\begin{tabular}[t]{lrrr}
\toprule
rowname & Linear.NF & Quadratic.NF & Zonation\\
\midrule
Linear.NF & NA & 0.9211996 & 0.6241322\\
Quadratic.NF & 0.9211996 & NA & 0.6690281\\
Zonation & 0.6241322 & 0.6690281 & NA\\
\bottomrule
\end{tabular}
\end{table}

\begin{figure}
\centering
\includegraphics{NFPaper_files/figure-latex/LocalCorr-1.png}
\caption{\label{fig:LocalCorr}Box plot of local correlations between all three different methods of priorization}
\end{figure}

\begin{figure}
\centering
\includegraphics{NFPaper_files/figure-latex/Boxplot-1.pdf}
\caption{\label{fig:Boxplot}Box plot of local correlations between all three different methods of priorization}
\end{figure}

\hypertarget{discussion}{%
\section{Discussion}\label{discussion}}

\hypertarget{primer-parrafo-comparison-to-other-methods-begin-by-revisiting-the-methodology-you-can-use-similar-language-or-even-similar-sentences-this-is-supposet-to-be-an-echo-to-prior-sentences}{%
\subsection{Primer parrafo (Comparison to other methods, begin by revisiting the methodology, you can use similar language or even similar sentences this is supposet to be an echo to prior sentences)}\label{primer-parrafo-comparison-to-other-methods-begin-by-revisiting-the-methodology-you-can-use-similar-language-or-even-similar-sentences-this-is-supposet-to-be-an-echo-to-prior-sentences}}

In this study we compared the solution of Quadratic Cost Network Flow, Linear Cost Network Flow and Zonation in order to solve for the conservation of 671 species of Plants and Vertebrates and compared the results of those solutions.

\hypertarget{segundo-parrafo}{%
\subsection{segundo parrafo}\label{segundo-parrafo}}

\hypertarget{section}{%
\subsubsection{2}\label{section}}

Other methods for Systematic Conservation Prioritization have either used just a few species, been limited to an area smaller than a country, not dealt with climate change, or only used the present and one future layer in order to solve this priorization problems, this last point might lead to connectivity problems.

\hypertarget{state-how-this-study-improves-on-this-or-how-it-fills-those-gaps}{%
\subsubsection{3 (State how this study improves on this or how it fills those gaps)}\label{state-how-this-study-improves-on-this-or-how-it-fills-those-gaps}}

In this study, we solved a problem for 671 species, in an area of 3,632,762 \(Km^2\) encompasing three countries and a buffer zone within their neighbouring countries. We used 8 decadal time-slices in order to ensure that the species can move form one conservation area to the ones in the future. We also compared between different methods using Spearman correlation

\hypertarget{tercer-parrafo}{%
\subsection{Tercer parrafo}\label{tercer-parrafo}}

\hypertarget{section-1}{%
\subsubsection{4}\label{section-1}}

In all three algorithms (Linear Network Flow, Quadratic Network Flow, and Zonation) have a high level of concordance in terms of which areas are important as shown by figures \ref{fig:AllSols}, \ref{fig:Binary} and \ref{fig:NumberOfTimes}

\hypertarget{section-2}{%
\subsubsection{5}\label{section-2}}

Shows how results enhances prior studies

\hypertarget{section-3}{%
\subsubsection{6}\label{section-3}}

Highlights some results of the study on it's own

\hypertarget{section-4}{%
\subsubsection{7}\label{section-4}}

Concludes on those results

\hypertarget{cuarto-parrafo}{%
\subsection{Cuarto parrafo}\label{cuarto-parrafo}}

\hypertarget{section-5}{%
\subsubsection{8}\label{section-5}}

Highlight the general study

\hypertarget{section-6}{%
\subsubsection{9}\label{section-6}}

Highlights the application

\hypertarget{section-7}{%
\subsubsection{10}\label{section-7}}

States that there might be some limitations

\hypertarget{quinto-parrafo}{%
\subsection{Quinto parrafo}\label{quinto-parrafo}}

\hypertarget{section-8}{%
\subsubsection{11}\label{section-8}}

Shows a general limitation of the research

\hypertarget{section-9}{%
\subsubsection{12}\label{section-9}}

States gaps of this research

\hypertarget{acknowledgments}{%
\section*{Acknowledgments}\label{acknowledgments}}
\addcontentsline{toc}{section}{Acknowledgments}

Funding:

\hypertarget{supplemental-data}{%
\section{Supplemental Data}\label{supplemental-data}}

Supplementary Material should be uploaded separately on submission, if there are
Supplementary Figures, please include the caption in the same file as the
figure. LaTeX Supplementary Material templates can be found in the Frontiers
LaTeX folder

\hypertarget{references}{%
\section{References}\label{references}}

\hypertarget{figures}{%
\section*{Figures}\label{figures}}
\addcontentsline{toc}{section}{Figures}

\hypertarget{suplmentary-materials}{%
\section{Suplmentary materials}\label{suplmentary-materials}}

\hypertarget{list-of-species}{%
\subsection{List of species}\label{list-of-species}}

The species used for the solution were the folowing:

Aa maderoi, Abutilon ibarrense, Achyrocline crassiceps, Aciachne flagellifera, Aechmea drakeana, Aegiphila pennellii, Aequatorium sinuatifolium, Aequatorium verrucosum, Aetheolaena involucrata, Aetheolaena lingulata, Ageratina asclepiadea, Ageratina dendroides, Ageratina gynoxoides, Ageratina pseudochilca, Ageratina theaefolia, Ageratina vacciniaefolia, Agouticarpa grandistipula, Agrostis boyacensis, Agrostis foliata, Agrostis trichodes, Aiphanes grandis, Altensteinia virescens, Amazilia castaneiventris, Anagallis foemina, Anastrophyllum nigrescens, Andigena laminirostris, Anomobryum prostratum, Anotomys leander, Anredera brachystachys, Anthopterus schultzeae, Anthurium aristatum, Anthurium carchiense, Anthurium cordiforme, Anthurium flavolineatum, Anthurium giganteum, Anthurium gualeanum, Anthurium herthae, Anthurium lancea, Anthurium membranaceum, Anthurium morae, Anthurium palenquense, Anthurium pallatangense, Anthurium pedunculare, Anthurium pendulispadix, Anthurium phyllobaris, Anthurium rimbachii, Anthurium rugulosum, Anthurium wattii, Aphanactis jamesoniana, Aphanactis ollgaardii, Aphanactis piloselloides, Aragoa abietina, Aragoa cundinamarcensis, Aragoa occidentalis, Arcytophyllum aristatum, Arenaria venezuelana, Aristeguietia glutinosa, Arracacia moschata, Arremon phaeopleurus, Astragalus geminiflorus, Astragalus sprucei, Atlapetes flaviceps, Atlapetes leucopis, Atractus crassicaudatus, Atractus lasallei, Axinaea merianiae, Axinaea sodiroi, Azorella aretioides, Azorella cuatrecasasii, Azorella pedunculata, Baccharis revoluta, Baccharis rupicola, Baccharis teindalensis, Badilloa salicina, Bangsia aureocincta, Bangsia edwardsi, Bangsia rothschildi, Bartsia laticrenata, Bartsia orthocarpiflora, Bartsia santolinifolia, Bartsia stricta, Bauhinia stenantha, Begonia tiliifolia, Belloa radians, Berberis goudotii, Berberis hallii, Berberis rigidifolia, Berberis verticillata, Besleria angustiflora, Blakea glandulosa, Blakea hispida, Blakea oldemanii, Blakea rotundifolia, Bolborhynchus ferrugineifrons, Bomarea glaucescens, Bomarea linifolia, Bomarea patacocensis, Bomarea uncifolia, Brachtia andina, Brachyotum alpinum, Brachyotum confertum, Brachyotum fictum, Brachyotum gleasonii, Brachyotum harlingii, Brachyotum jamesonii, Brachyotum strigosum, Brayopsis colombiana, Breutelia squarrosa, Breutelia trianae, Browneopsis disepala, Brugmansia aurea, Brunellia boqueronensis, Brunellia colombiana, Bucquetia glutinosa, Buddleja pichinchensis, Burmeistera montipomum, Calamagrostis aurea, Calamagrostis effusa, Calamagrostis fibrovaginata, Calamagrostis guamanensis, Calamagrostis mollis, Calamagrostis podophora, Calceolaria dilatata, Calceolaria ferruginea, Calceolaria hyssopifolia, Calceolaria lamiifolia, Calceolaria pedunculata, Calceolaria purpurascens, Calceolaria rosmarinifolia, Calliergonella cuspidata, Campylopus argyrocaulon, Campylopus cleefii, Campylopus cucullatifolius, Campylopus edithae, Campylopus incertus, Campylopus pittieri, Campylopus subconcolor, Carex amicta, Casearia mexiae, Castilleja nubigena, Castratella piloselloides, Centradeniastrum album, Centropogon aequatorialis, Centropogon baezanus, Centropogon glabrifilis, Centropogon medusa, Centropogon nigricans, Centropogon subandinus, Cerastium floccosum, Ceratostema alatum, Cestrum peruvianum, Chalcostigma heteropogon, Chorisodontium speciosum, Chromolaena bullata, Chusquea subulata, Cinclodes excelsior, Cistothorus apolinari, Cistothorus meridae, Citharexylum sulcatum, Clavija eggersiana, Clethra crispa, Clinopodium nubigenum, Clinopodium taxifolium, Clinopodium tomentosum, Clusia callejasii, Coespeletia moritziana, Coespeletia spicata, Coespeletia thyrsiformis, Coespeletia timotensis, Colignonia pentoptera, Columnea albiflora, Columnea capillosa, Columnea ciliata, Columnea eubracteata, Conirostrum rufum, Conyza cardaminifolia, Conyza trihecatactis, Coursetia gracilis, Coussarea pilosiflora, Cranichis antioquiensis, Cranioleuca hellmayri, Critoniopsis occidentalis, Critoniopsis palaciosii, Critoniopsis sodiroi, Cronquistianthus niveus, Croton elegans, Croton rimbachii, Cryptotis equatoris, Culcitium nivale, Cyanolyca pulchra, Cynanchum microphyllum, Dalea humifusa, Dasyphyllum popayanense, Dendrophorbium lloense, Diglossa gloriosa, Diglossa gloriosissima, Diglossa indigotica, Diplostephium alveolatum, Diplostephium bicolor, Diplostephium cayambense, Diplostephium cinerascens, Diplostephium ericoides, Diplostephium frontinense, Diplostephium glandulosum, Diplostephium hartwegii, Diplostephium ochraceum, Diplostephium phylicoides, Diplostephium revolutum, Diplostephium rhododendroides, Diplostephium rhomboidale, Diplostephium rupestre, Diplostephium schultzii, Diplostephium spinulosum, Diplostephium tenuifolium, Diplostephium tolimense, Diplostephium violaceum, Diplostichum longirostre, Distichia acicularis, Ditrichum submersum, Draba aretioides, Draba hallii, Draba obovata, Draba splendens, Draba spruceana, Dracula sodiroi, Drepanocladus exannulatus, Drymonia crenatiloba, Drymonia laciniosa, Dysithamnus occidentalis, Elaphoglossum antisanae, Elaphoglossum heliconiifolium, Elaphoglossum yatesii, Elasis hirsuta, Elatine ecuadoriensis, Elleanthus gastroglottis, Elleanthus petrogeiton, Epidendrum englerianum, Epidendrum marsupiale, Epidendrum pallatangae, Epidendrum pichinchae, Epidendrum porphyreum, Eragrostis condensata, Erigeron ecuadoriensis, Eriocnemis cupreoventris, Eriocnemis derbyi, Eriocnemis mosquera, Eryngium humboldtii, Espeletia argentea, Espeletia barclayana, Espeletia batata, Espeletia boyacensis, Espeletia brassicoidea, Espeletia cleefii, Espeletia congestiflora, Espeletia conglomerata, Espeletia frontinoensis, Espeletia grandiflora, Espeletia hartwegiana, Espeletia killipii, Espeletia lopezii, Espeletia murilloi, Espeletia pycnophylla, Espeletia schultesiana, Espeletia schultzii, Espeletia semiglobulata, Espeletia summapacis, Espeletia tunjana, Espeletia uribei, Espeletiopsis colombiana, Espeletiopsis corymbosa, Espeletiopsis pannosa, Espeletiopsis petiolata, Eudema nubigena, Euphonia concinna, Festuca andicola, Festuca chimborazensis, Festuca glumosa, Festuca subulifolia, Festuca vaginalis, Fuchsia ampliata, Fuchsia caucana, Fuchsia dependens, Fuchsia hartwegii, Fuchsia macrostigma, Fuchsia orientalis, Fuchsia sessilifolia, Fuchsia sylvatica, Fuchsia vulcanica, Gasteranthus columbianus, Gasteranthus lateralis, Gasteranthus leopardus, Gasteranthus quitensis, Gaultheria cordifolia, Geissanthus argutus, Genista monspessulana, Gentianella cerastioides, Gentianella cernua, Gentianella corymbosa, Gentianella foliosa, Gentianella hirculus, Gentianella hyssopifolia, Gentianella nummulariifolia, Gentianella rapunculoides, Gentianella rupicola, Geonoma divisa, Geranium maniculatum, Geranium multiceps, Geranium multipartitum, Glaucidium nubicola, Gongylolepis jauaensis, Grallaria alleni, Grallaria flavotincta, Grallaria milleri, Grallaria rufocinerea, Grallaricula lineifrons, Grallaricula loricata, Grosvenoria hypargyra, Grosvenoria rimbachii, Gustavia longifuniculata, Gustavia serrata, Guzmania jaramilloi, Guzmania lehmanniana, Guzmania vanvolxemii, Guzmania wittmackii, Gynoxys acostae, Gynoxys hirsuta, Gynoxys laurata, Gynoxys miniphylla, Gynoxys parvifolia, Gynoxys pendula, Gynoxys sodiroi, Gynoxys stuebelii, Gynoxys tolimensis, Halenia adpressa, Halenia gentianoides, Halenia pulchella, Haplophaedia lugens, Hebeclinium tetragonum, Hedyosmum parvifolium, Heliangelus strophianus, Heliconia dielsiana, Heppiella verticillata, Hesperomeles goudotiana, Hieracium avilae, Huperzia capellae, Huperzia cruenta, Huperzia cumingii, Huperzia hystrix, Huperzia lindenii, Huperzia llanganatensis, Huperzia polydactyla, Huperzia rufescens, Huperzia transilla, Huperzia unguiculata, Hypericum decandrum, Hypericum goyanesii, Hypericum myricariifolium, Hypericum prostratum, Hypericum quitense, Hypochaeris setosa, Inga extra-nodis, Isoetes andina, Isoetes bischlerae, Isoetes boyacensis, Isoetes novo-granadensis, Isoetes palmeri, Jamesonia bogotensis, Jamesonia cinnamomea, Jamesonia pulchra, Juncus breviculmis, Juncus echinocephalus, Juncus ecuadoriensis, Lachemilla angustata, Lachemilla hirta, Lachemilla hispidula, Lachemilla holosericea, Lachemilla jamesonii, Lachemilla nivalis, Lachemilla rupestris, Lachemilla tanacetifolia, Lasiocephalus ovatus, Ledothamnus luteus, Lepanthes effusa, Lepanthes pilosella, Lepechinia vulcanicola, Lepidozia macrocolea, Leptodontium erythroneuron, Leptoscyphus cleefii, Leptotila conoveri, Liabum saloyense, Llerasia hypoleuca, Loricaria antisanensis, Loricaria complanata, Loricaria ilinissae, Lourteigia humilis, Lourteigia microphylla, Lupinus alopecuroides, Lupinus colombiensis, Lupinus smithianus, Lysipomia montioides, Lysipomia muscoides, Macleania bullata, Macleania coccoloboides, Macleania cordifolia, Macleania ericae, Macleania loeseneriana, Macrocarpaea glabra, Margarornis stellatus, Masdevallia ophioglossa, Matisia palenquiana, Mauritiella macroclada, Melothria longituba, Melpomene assurgens, Melpomene sodiroi, Meriania heptamera, Meriania hernandoi, Metallura baroni, Metallura williami, Mezobromelia lyman-smithii, Miconia aguirrei, Miconia biappendiculata, Miconia chlorocarpa, Miconia dielsii, Miconia glandulistyla, Miconia hymenanthera, Miconia nodosa, Miconia pernettifolia, Miconia pichinchensis, Miconia psychrophila, Miconia scutata, Miconia summa, Miconia turgida, Mimosa quitensis, Mimosa townsendii, Mona meridensis, Monnina crassifolia, Monnina equatoriensis, Monnina obtusifolia, Monnina pseudopilosa, Monnina revoluta, Monnina sodiroana, Monnina tenuifolia, Monticalia arbutifolia, Monticalia peruviana, Monticalia vaccinioides, Morella singularis, Muscisaxicola alpinus, Mutisia grandiflora, Mutisia microphylla, Mutisia sodiroi, Myioborus albifrons, Myrrhidendron glaucescens, Myrsine panamensis, Niphogeton azorelloides, Niphogeton glaucescens, Niphogeton josei, Notopleura marginata, Nototriche jamesonii, Nototriche phyllanthos, Odontoglossum hallii, Odontophorus atrifrons, Odontophorus melanonotus, Oncidium serratum, Opuntia soederstromiana, Oreobolus cleefii, Oreopanax corazonensis, Oreopanax ecuadorensis, Oreopanax ellsworthii, Oreopanax grandifolius, Oreopanax mutisianus, Oreopanax nigrum, Oreopanax palamophyllus, Oreopanax ruizanus, Oreothraupis arremonops, Oreotrochilus chimborazo, Ossaea sessilifolia, Oxypogon guerinii, Paepalanthus alpinus, Paepalanthus karstenii, Paepalanthus lodiculoides, Palicourea prodiga, Palicourea sodiroi, Palicourea subalatoides, Pappobolus imbaburensis, Paspalum hirtum, Passiflora adulterina, Passiflora crispolanata, Passiflora cuatrecasasii, Passiflora jamesonii, Passiflora roseorum, Passiflora trinervia, Pedicularis incurva, Pentacalia abietina, Pentacalia americana, Pentacalia arbutifolia, Pentacalia flosfragrans, Pentacalia guadalupe, Pentacalia ledifolia, Pentacalia nitida, Pentacalia pulchella, Pentacalia reissiana, Pentacalia vaccinioides, Pentacalia vernicosa, Pentagonia grandiflora, Peperomia fuscipunctata, Peperomia miqueliana, Pernettya hirta, Phalcoboenus carunculatus, Philodendron oligospermum, Philodendron rugosum, Philonotis andina, Pilea gallowayana, Pinguicula elongata, Piper laguna-cochanum, Piper sodiroi, Piper subflavum, Pipreola jucunda, Pitcairnia commixta, Pitcairnia cosangaensis, Pitcairnia fusca, Pittosporum undulatum, Plagiocheilus frigidus, Plagiocheilus solivaeformis, Pleurothallis ensata, Pleurothallis macra, Pleurothallis truncata, Plutarchia guascensis, Poa cucullata, Poa paramoensis, Poa pauciflora, Polylepis lanuginosa, Polylepis quadrijuga, Polylepis reticulata, Polypodium mindense, Polypodium monosorum, Polypodium segregatum, Polytrichadelphus ciliatus, Pouteria capacifolia, Prunus antioquensis, Prunus herthae, Psammisia debilis, Psammisia occidentalis, Psammisia sclerantha, Psychotria boqueronensis, Psychotria cuatrecasasii, Puya glomerifera, Puya goudotiana, Puya nitida, Puya santosii, Puya trianae, Pyrrhura hoematotis, Racinaea subalata, Rallus semiplumbeus, Ranunculus nubigenus, Renealmia fragilis, Renealmia sessilifolia, Rhodospatha densinervia, Rhodostemonodaphne laxa, Rhynchospora paramora, Ribes hirtum, Ribes lehmannii, Rubus choachiensis, Rubus gachetensis, Ruilopezia atropurpurea, Ruilopezia marcescens, Rumex tolimensis, Saguinus leucopus, Salvia pichinchensis, Saurauia chiliantha, Saurauia crassisepala, Saurauia omichlophila, Saurauia peduncularis, Saurauia pseudostrigillosa, Schefflera acaropunctata, Schefflera lasiogyne, Schefflera marginata, Schefflera ramosissima, Scorpidium scorpioides, Scytalopus griseicollis, Scytalopus rodriguezi, Scytalopus vicinior, Senecio culcitioides, Senecio formosoides, Senecio hypsobates, Senecio leucanthemoides, Senecio madagascariensis, Senecio niveoaureus, Senecio subruncinatus, Setaria cernua, Siparuna croatii, Siparuna palenquensis, Siphocampylus affinis, Siphocampylus brevicalyx, Siphocampylus columnae, Solanum crinitipes, Solanum paucijugum, Sorocea sarcocarpa, Sparrmannia africana, Sphaeradenia hamata, Sphaeradenia horrida, Sphagnum compactum, Sphagnum oxyphyllum, Sphyrospermum boekei, Sphyrospermum grandifolium, Sporobolus bogotensis, Stachys bogotensis, Stachys elliptica, Stelis columnaris, Stelis piperina, Stellaria recurvata, Stenospermation longifolium, Stilpnophyllum grandifolium, Stilpnophyllum oellgaardii, Stipa milleana, Stipa rosea, Symphyogyna bogotensis, Symplocos cundinamarcensis, Symplocos rhomboidea, Symplocos theiformis, Synallaxis castanea, Synallaxis subpudica, Syntrichia andicola, Tagetes zypaquirensis, Tangara rufigenis, Telipogon nervosus, Ternstroemia cleistogama, Terpsichore heteromorpha, Terpsichore pichinchae, Tetragastris varians, Thelypteris rigescens, Themistoclesia epiphytica, Theobroma gileri, Thibaudia andrei, Thibaudia grantii, Thibaudia parvifolia, Thomasomys hylophilus, Thomasomys niveipes, Thryophilus nicefori, Tibouchina gleasoniana, Tibouchina paleacea, Tibouchina pendula, Tillandsia emergens, Tillandsia incarnata, Tillandsia lajensis, Tillandsia secunda, Tillandsia superba, Tillandsia truncata, Tournefortia ramosissima, Uncinia paludosa, Uniola condensata, Urothraupis stolzmanni, Urtica longispica, Valeriana adscendens, Valeriana arborea, Valeriana bracteata, Valeriana tatamana, Valeriana tomentosa, Verbesina brachypoda, Verbesina latisquama, Verbesina lloensis, Verbesina sodiroi, Viburnum anabaptista, Viburnum antioquiense, Vigna venusta, Viola glandularis, Vireo masteri, Weinmannia polyphylla, Wettinia anomala, Xenophyllum crassum, Zanthoxylum lepidopteriphilum, Zygodon pichinchensis

\hypertarget{refs}{}
\leavevmode\hypertarget{ref-Ahuja93}{}%
Ahuja, R. K., Magnanti, T. L., \& Orlin, J. B. (1993). \emph{Network flows: Theory, algorithms, and applications}. Englewood Cliffs, NJ: Prentice Hall.

\leavevmode\hypertarget{ref-alagador2014shifting}{}%
Alagador, D., Cerdeira, J. O., \& Araújo, M. B. (2014). Shifting protected areas: Scheduling spatial priorities under climate change. \emph{Journal of Applied Ecology}, \emph{51}(3), 703--713.

\leavevmode\hypertarget{ref-araujo2004would}{}%
Araújo, M. B., Cabeza, M., Thuiller, W., Hannah, L., \& Williams, P. H. (2004). Would climate change drive species out of reserves? An assessment of existing reserve-selection methods. \emph{Global Change Biology}, \emph{10}(9), 1618--1626.

\leavevmode\hypertarget{ref-beier2007linkage}{}%
Beier, P., Majka, D., \& Bayless, T. (2007). Linkage designs for arizona's missing linkages. \emph{Arizona Game and Fish Department, Phoenix}.

\leavevmode\hypertarget{ref-Boyle2019GNRS}{}%
Boyle, B. (2019). Geographic name resolution service (gnrs). \emph{GitHub repository}. \url{https://github.com/ojalaquellueva/gnrs}; GitHub.

\leavevmode\hypertarget{ref-boyle2013taxonomic}{}%
Boyle, B., Hopkins, N., Lu, Z., Garay, J. A. R., Mozzherin, D., Rees, T., \ldots{} others. (2013). The taxonomic name resolution service: An online tool for automated standardization of plant names. \emph{BMC Bioinformatics}, \emph{14}(1), 16.

\leavevmode\hypertarget{ref-chen2011rapid}{}%
Chen, I.-C., Hill, J. K., Ohlemüller, R., Roy, D. B., \& Thomas, C. D. (2011). Rapid range shifts of species associated with high levels of climate warming. \emph{Science}, \emph{333}(6045), 1024--1026.

\leavevmode\hypertarget{ref-Corcoran_Quadcost}{}%
Corcoran, D., \& Fajardo, J. (2019). \emph{QuadCostAmpl: Transforms a stack of timeslices of species distribution models into data for ampl modeling}. Retrieved from \url{https://github.com/derek-corcoran-barrios/QuadCostAmpl}

\leavevmode\hypertarget{ref-cushman2013biological}{}%
Cushman, S. A., McRae, B., Adriaensen, F., Beier, P., Shirley, M., \& Zeller, K. (2013). Biological corridors and connectivity {[}chapter 21{]}. \emph{In: Macdonald, DW; Willis, KJ, Eds. Key Topics in Conservation Biology 2. Hoboken, NJ: Wiley-Blackwell. P. 384-404.}, 384--404.

\leavevmode\hypertarget{ref-de2018soilgrids}{}%
De Sousa, L., Heuvelink, G., Batjes, N., \& Kempen, B. (2018). SoilGrids: Using big data solutions and machine learning algorithms for global soil mapping. In \emph{Scientific symposium fair data sciences for green life sciences} (pp. 1--1).

\leavevmode\hypertarget{ref-di2014quick}{}%
Di Minin, E., Veach, V., Lehtomäki, J., Montesino Pouzols, F., Moilanen, A., \& others. (2014). A quick introduction to zonation.

\leavevmode\hypertarget{ref-fajardo_javier_2018_2669407}{}%
Fajardo, J., Corcoran, D., Roehrdanz, P., Hannah, L., \& Marquet, P. (2018, October). doi:\href{https://doi.org/10.5281/zenodo.2669407}{10.5281/zenodo.2669407}

\leavevmode\hypertarget{ref-gregory2014forecasts}{}%
Gregory, S. D., Ancrenaz, M., Brook, B. W., Goossens, B., Alfred, R., Ambu, L. N., \& Fordham, D. A. (2014). Forecasts of habitat suitability improve habitat corridor efficacy in rapidly changing environments. \emph{Diversity and Distributions}, \emph{20}(9), 1044--1057.

\leavevmode\hypertarget{ref-hannah2007protected}{}%
Hannah, L., Midgley, G., Andelman, S., Araújo, M., Hughes, G., Martinez-Meyer, E., \ldots{} Williams, P. (2007). Protected area needs in a changing climate. \emph{Frontiers in Ecology and the Environment}, \emph{5}(3), 131--138.

\leavevmode\hypertarget{ref-Hanson2019}{}%
Hanson, J. O., Schuster, R., Morrell, N., Strimas-Mackey, M., Watts, M. E., Arcese, P., \ldots{} Possingham, H. P. (2019). \emph{Prioritizr: Systematic conservation prioritization in r}. Retrieved from \url{https://CRAN.R-project.org/package=prioritizr}

\leavevmode\hypertarget{ref-Hijmans_Dismo}{}%
Hijmans, R. J., Phillips, S., Leathwick, J., \& Elith, J. (2017). \emph{Dismo: Species distribution modeling}. Retrieved from \url{https://CRAN.R-project.org/package=dismo}

\leavevmode\hypertarget{ref-jenkins2015us}{}%
Jenkins, C. N., Van Houtan, K. S., Pimm, S. L., \& Sexton, J. O. (2015). US protected lands mismatch biodiversity priorities. \emph{Proceedings of the National Academy of Sciences}, \emph{112}(16), 5081--5086.

\leavevmode\hypertarget{ref-jones2016incorporating}{}%
Jones, K. R., Watson, J. E., Possingham, H. P., \& Klein, C. J. (2016). Incorporating climate change into spatial conservation prioritisation: A review. \emph{Biological Conservation}, \emph{194}, 121--130.

\leavevmode\hypertarget{ref-lawler2013projected}{}%
Lawler, J., Ruesch, A., Olden, J., \& McRae, B. (2013). Projected climate-driven faunal movement routes. \emph{Ecology Letters}, \emph{16}(8), 1014--1022.

\leavevmode\hypertarget{ref-lenoir2008significant}{}%
Lenoir, J., Gégout, J.-C., Marquet, P., De Ruffray, P., \& Brisse, H. (2008). A significant upward shift in plant species optimum elevation during the 20th century. \emph{Science}, \emph{320}(5884), 1768--1771.

\leavevmode\hypertarget{ref-marquet_2019}{}%
Marquet, P., Lessman, J., \& Shaw, R. (2019). Protected-area management and climate change. In T. E. Lovejoy, L. Hannah, \& E. O. Wilson (Eds.), \emph{Biodiversity and climate change: Transforming the biosphere} (pp. 283--293). New Haven, Connecticut: Yale University Press.

\leavevmode\hypertarget{ref-merow2014we}{}%
Merow, C., Smith, M. J., Edwards Jr, T. C., Guisan, A., McMahon, S. M., Normand, S., \ldots{} Elith, J. (2014). What do we gain from simplicity versus complexity in species distribution models? \emph{Ecography}, \emph{37}(12), 1267--1281.

\leavevmode\hypertarget{ref-merow2013practical}{}%
Merow, C., Smith, M. J., \& Silander Jr, J. A. (2013). A practical guide to maxent for modeling species' distributions: What it does, and why inputs and settings matter. \emph{Ecography}, \emph{36}(10), 1058--1069.

\leavevmode\hypertarget{ref-naidoo2007global}{}%
Naidoo, R., \& Iwamura, T. (2007). Global-scale mapping of economic benefits from agricultural lands: Implications for conservation priorities. \emph{Biological Conservation}, \emph{140}(1-2), 40--49.

\leavevmode\hypertarget{ref-nunez2013connectivity}{}%
Nuñez, T. A., Lawler, J. J., McRae, B. H., Pierce, D. J., Krosby, M. B., Kavanagh, D. M., \ldots{} Tewksbury, J. J. (2013). Connectivity planning to address climate change. \emph{Conservation Biology}, \emph{27}(2), 407--416.

\leavevmode\hypertarget{ref-phillips2006maximum}{}%
Phillips, S. J., Anderson, R. P., \& Schapire, R. E. (2006). Maximum entropy modeling of species geographic distributions. \emph{Ecological Modelling}, \emph{190}(3-4), 231--259.

\leavevmode\hypertarget{ref-phillips2008optimizing}{}%
Phillips, S. J., Williams, P., Midgley, G., \& Archer, A. (2008). Optimizing dispersal corridors for the cape proteaceae using network flow. \emph{Ecological Applications}, \emph{18}(5), 1200--1211.

\leavevmode\hypertarget{ref-regos2016predicting}{}%
Regos, A., D'Amen, M., Titeux, N., Herrando, S., Guisan, A., \& Brotons, L. (2016). Predicting the future effectiveness of protected areas for bird conservation in mediterranean ecosystems under climate change and novel fire regime scenarios. \emph{Diversity and Distributions}, \emph{22}(1), 83--96.

\leavevmode\hypertarget{ref-reside2017trade}{}%
Reside, A. E., VanDerWal, J., \& Moran, C. (2017). Trade-offs in carbon storage and biodiversity conservation under climate change reveal risk to endemic species. \emph{Biological Conservation}, \emph{207}, 9--16.

\leavevmode\hypertarget{ref-rodrigues2004effectiveness}{}%
Rodrigues, A. S., Andelman, S. J., Bakarr, M. I., Boitani, L., Brooks, T. M., Cowling, R. M., \ldots{} others. (2004). Effectiveness of the global protected area network in representing species diversity. \emph{Nature}, \emph{428}(6983), 640.

\leavevmode\hypertarget{ref-rosenberg1997biological}{}%
Rosenberg, D. K., Noon, B. R., \& Meslow, E. C. (1997). Biological corridors: Form, function, and efficacy. \emph{BioScience}, \emph{47}(10), 677--687.

\leavevmode\hypertarget{ref-tuanmu2014global}{}%
Tuanmu, M.-N., \& Jetz, W. (2014). A global 1-km consensus land-cover product for biodiversity and ecosystem modelling. \emph{Global Ecology and Biogeography}, \emph{23}(9), 1031--1045.

\leavevmode\hypertarget{ref-unep2018ngs}{}%
UNEP-WCMC, I. (2018). NGS (2018). \emph{Protected Planet Report}.

\leavevmode\hypertarget{ref-williams2005planning}{}%
Williams, P., Hannah, L., Andelman, S., Midgley, G., Araújo, M., Hughes, G., \ldots{} Pearson, R. (2005). Planning for climate change: Identifying minimum-dispersal corridors for the cape proteaceae. \emph{Conservation Biology}, \emph{19}(4), 1063--1074.

\leavevmode\hypertarget{ref-zappa2017storylines}{}%
Zappa, G., \& Shepherd, T. G. (2017). Storylines of atmospheric circulation change for european regional climate impact assessment. \emph{Journal of Climate}, \emph{30}(16), 6561--6577.

\leavevmode\hypertarget{ref-zar1972significance}{}%
Zar, J. H. (1972). Significance testing of the spearman rank correlation coefficient. \emph{Journal of the American Statistical Association}, \emph{67}(339), 578--580.


\end{document}
